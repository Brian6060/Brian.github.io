\documentclass[12pt]{ctexart}
\usepackage[margin=2.5cm]{geometry}
\usepackage{graphicx}
\usepackage{float}
\usepackage{hyperref}
\usepackage{array}
\usepackage{longtable}
\usepackage{booktabs}
\usepackage{caption}
\graphicspath{{media/}}
\setlength{\parskip}{0.6em}
\setlength{\parindent}{0pt}
\begin{document}




大学生创新训练计划

创新类项目申报书

项目名称:   基于粒球图表示学习的医学图像异常检测

项目负责人:                杜博源

所在学院:计算机学院(软件学院、智能科学与技术学院)

专业年级:               2025级

学    号:           2024151470021

手    机:            17800886726

电子邮箱:   2024151470021@stu.scu.edu.cn

指导教师:                袁钟

项目起止年月:   2024年11月至2025年10月

项目参与学生人数:

四川大学教务处制

年    月

填写说明

一、凡申报四川大学“大学生创新训练计划”必须填写本申报书。创新类项目是本科生个人或团队,在导师指导下,自主完成创新性研究项目设计、研究条件准备和项目实施、研究报告撰写、成果(学术)交流等工作。

二、“项目所属一级学科和代码”参考《普通高等学校本科专业目录和专业介绍(2012年)》。

三、“项目开展支撑平台”指支撑本项目开展的国家级和省部级重点实验室(中心、平台等)、国家双创示范基地平台、教学实验中心(实验室)、企业、事业或其他单位等,表中填写平台名称,可以多个。

四、“项目组成员”人数原则上不超过五人,应排序。

五、“项目成熟度”请参考附件《项目成熟度量表》。

六、本书应该填写完整、内容详实、表达准确,数字一律填写阿拉伯数字。

七、报送申报书的电子文档至负责人所在学院。

\textbackslash{}\textbackslash{}

\begin{table}[H]
\centering
\begin{tabular}{|l|l|l|l|l|}
\hline
 项目名称  \tabularnewline  基于粒球图表示学习的医学图像异常诊断技术研究  \tabularnewline     \tabularnewline     \tabularnewline     \\ \hline
 项目属性  \tabularnewline  面上项目 “人工智能+新质战略育苗”(含2035特区子计划项目)
交叉学科子计划项目  \tabularnewline     \tabularnewline     \tabularnewline     \\ \hline
 申请类别  \tabularnewline  科学探索与工程技术类 人文艺术与社会科学类 
软件信息与文创类     智能装备与医疗器械类  
生物医药与新材料类  \tabularnewline     \tabularnewline     \tabularnewline     \\ \hline
 申请经费  \tabularnewline  元  \tabularnewline  起止时间  \tabularnewline  2024年11月至2025年10月  \tabularnewline     \\ \hline
 项目所属     一级学科和代码  \tabularnewline     \tabularnewline     \tabularnewline     \tabularnewline     \\ \hline
 项目开展                支撑平台  \tabularnewline     \tabularnewline     \tabularnewline     \tabularnewline     \\ \hline
 项目来源
(可多选)  \tabularnewline  十大重点支持领域的项目
进课题组、进实验室、进科研团队参与的项目
国家级和省部级重点实验室(中心、平台等)、国家双创示范基地平台支持申报项目
交叉学科创新项目
“青年红色筑梦之旅”计划项目
基于前期研究实践成果、继续深入研究实践的创新项目
高水平课题
其他  \tabularnewline     \tabularnewline     \tabularnewline     \\ \hline
 高水平课题名称(非高水平课题可不填)  \tabularnewline  命题名称  \tabularnewline     \tabularnewline     \tabularnewline     \\ \hline
    \tabularnewline  校内指导老师姓名(非交叉学科子计划项目一般仅允许一位指导老师)  \tabularnewline     \tabularnewline     \tabularnewline     \\ \hline
 所属重点支持领域(可不选)  \tabularnewline  选择1项:
A.不填
B.泛终端芯片及操作系统应用开发
C.重大应用关键软件
D.云计算和大数据
E.人工智能
F.无人驾驶
G.新能源与储能技术
H.生物技术与生物育种
I.绿色环保与固废资源化
J.第五代通信技术和新一代IP网络通信技术
K.社会事业与文化传承  \tabularnewline     \tabularnewline     \tabularnewline     \\ \hline
 负责人之前参与大创项目情况  \tabularnewline  格式如下,有则填,不限条目:
1. 负责人/团队成员,项目编号,项目名称,立项级别,项目类别,立项年份,结题成绩;
2. ……  \tabularnewline     \tabularnewline     \tabularnewline     \\ \hline
 项目成员之前参与大创项目情况  \tabularnewline  1. 姓名,负责人/团队成员,项目编号,项目名称,立项级别,项目类别,立项年份,结题成绩;
2. ……  \tabularnewline     \tabularnewline     \tabularnewline     \\ \hline
 项目负责人基本信息  \tabularnewline     \tabularnewline     \tabularnewline     \tabularnewline     \\ \hline
 姓名  \tabularnewline  学号  \tabularnewline  专业年级  \tabularnewline  所在学院  \tabularnewline     \\ \hline
    \tabularnewline     \tabularnewline     \tabularnewline     \tabularnewline     \\ \hline
 性别  \tabularnewline  手机  \tabularnewline  电子邮箱  \tabularnewline  身份证号  \tabularnewline     \\ \hline
    \tabularnewline     \tabularnewline     \tabularnewline     \tabularnewline     \\ \hline
 项目组成员基本信息  \tabularnewline     \tabularnewline     \tabularnewline     \tabularnewline     \\ \hline
 序号(含排序)  \tabularnewline  1  \tabularnewline  2  \tabularnewline  3  \tabularnewline  4  \\ \hline
 姓名/性别  \tabularnewline     \tabularnewline     \tabularnewline     \tabularnewline     \\ \hline
 学号  \tabularnewline     \tabularnewline     \tabularnewline     \tabularnewline     \\ \hline
 专业年级  \tabularnewline     \tabularnewline     \tabularnewline     \tabularnewline     \\ \hline
 所在学院  \tabularnewline     \tabularnewline     \tabularnewline     \tabularnewline     \\ \hline
 手机  \tabularnewline     \tabularnewline     \tabularnewline     \tabularnewline     \\ \hline
 电子邮箱  \tabularnewline     \tabularnewline     \tabularnewline     \tabularnewline     \\ \hline
 身份证号  \tabularnewline     \tabularnewline     \tabularnewline     \tabularnewline     \\ \hline
 签名  \tabularnewline     \tabularnewline     \tabularnewline     \tabularnewline     \\ \hline
 指导教师1 基本信息
(非交叉学科子计划项目一般仅允许一位指导老师)  \tabularnewline     \tabularnewline     \tabularnewline     \tabularnewline     \\ \hline
 姓名  \tabularnewline  所在学院或单位  \tabularnewline  研究方向  \tabularnewline  职称/职务  \tabularnewline     \\ \hline
    \tabularnewline     \tabularnewline     \tabularnewline     \tabularnewline     \\ \hline
 性别/年龄  \tabularnewline  手机  \tabularnewline  电子邮箱  \tabularnewline  签名  \tabularnewline     \\ \hline
\end{tabular}
\end{table}

\begin{table}[H]
\centering
\begin{tabular}{|l|l|l|l|}
\hline
    \tabularnewline     \tabularnewline     \tabularnewline     \\ \hline
 指导教师2  基本信息
(交叉学科子计划项目需填写第二指导老师)  \tabularnewline     \tabularnewline     \tabularnewline     \\ \hline
 姓名  \tabularnewline  所在学院或单位  \tabularnewline  研究方向  \tabularnewline  职称/职务  \\ \hline
    \tabularnewline     \tabularnewline     \tabularnewline     \\ \hline
 性别/年龄  \tabularnewline  手机  \tabularnewline  电子邮箱  \tabularnewline  签名  \\ \hline
    \tabularnewline     \tabularnewline     \tabularnewline     \\ \hline
\end{tabular}
\end{table}

\begin{table}[H]
\centering
\begin{tabular}{|l|}
\hline
 项目摘要(限200字以内)  \\ \hline
 摘要一般包括五部分:研究领域背景和进展;亟待解决的问题;项目的科学假说;为证明该假说需要开展的研究;研究的意义。  \\ \hline
    \\ \hline
    \\ \hline
 特色创新点(限100字以内,建议2-3点)  \\ \hline
    \\ \hline
\end{tabular}
\end{table}

报告正文

\begin{table}[H]
\centering
\begin{tabular}{|l|}
\hline
 立项依据与研究内容(建议8000字以下(不包括文献))  \\ \hline
 1.项目的立项依据(【理工医科】研究意义、国内外研究现状及发展动态分析,需结合科学研究发展趋势来论述科学意义;或结合国民经济和社会发展中迫切需要解决的关键科技问题来论述其应用前景。附主要参考文献目录【哲学社会学科】国内外相关研究的学术史梳理及研究动态;本课题相对于已有研究的独到学术价值和应用价值等。附主要参考文献目录)  \\ \hline
 1. 项目背景及挑战
1.1 背景
在现代医学场景中,医学影像已成为临床检测诊断的最基础、最重要的根据和参考。随着现代医疗影像采集技术的发展,医学影像的样本基数、数据规模与分辨率持续攀升,例如CT、MRI、超声图像等。如今,医生阅片的负荷和难度显著升高,并且不同设备间的差异与噪声干扰也在不断扩大,在传统医学诊断的过程中极易引发漏诊、误诊以及结论不一致的后果。《中国医疗机构影像科现状调查报告(2024)》白皮书中明确指出:2024年,我国CT、MRI等大型影像设备检查量达6.8亿人次,同比增长12\%,而影像科医生数量仅增长3.5\%。三甲医院影像科医生日均工作时间超12小时,人均年阅片量达8.5万例,疲劳诊断导致的误诊率上升至8.2\%[1]。
如今,医学图像异常检测作为计算机辅助诊断的新技术之一,帮助医生提升了发现识别医学图像中可疑区域的效率和精准度。然而,在以深度学习网络和机器学习为基础的计算机视觉系统中,现有医学图像作为学习样本,在医学图像异常检测“检测—分割—诊断”的链条中仍具有很大缺陷:对于检测部分,医学图像中小病灶、微异常区域极易被淹没,其体积小、异常少、对比度低的特点导致检测算法中,图像背景噪声往往会将其覆盖;分割步骤中,许多病灶与临近组织灰度接近,边界与轮廓模糊,并且作为部分学习样本的人工勾边图像主观性强,边界一致性差;诊断过程中,医学图像异常检测方法是基于深度学习网络的算法,其本身具有端到端、过程不可解释的特性,在以病人作为对象的疾病诊断过程中,无法给出诊断结果的核查证据,在有限的训练样本条件下,泛化性较差。
《AI医学影像诊断现状报告》中指出,医学影像异常检测AI在胸部X光诊断中的误诊率约为3\%,而在乳腺超声诊断中,误诊率高达10\%[2]。对此,改进医学图像异常检测检测技术中的技术缺陷和不足、提高异常检测的效率及精准度、鲁棒性以及可解释性是在解放医疗生产力的同时提高检测与诊断的准确性的关键与重点,对于降低误诊风险具有重要意义,是医疗图像异常检测的优化和发展方向。
【图1.1】人工疲劳误诊率+医学图像异常检测误诊率官方数据图片
早期的图像分析主要在像素网格上做计算,后来为了减轻计算量与噪声干扰,出现了“超像素”等固定分块的方法,但它们的粒度不够灵活。学界随后提出“粒度化思维”,用更贴近数据结构的“颗粒”来表示信息;在此基础上,粒球计算把“相似像素自动抱团成球”的想法落到了实处:平坦区域自然合并,细节和边界自动细化。随着图学习的兴起,这一思路被发展为粒球图表示——先把图像(或神经网络特征图)转成由粒球组成的图,再用GNN、Transformer等深度神经网络进行学习。近年,它因同时兼顾效率、精度与可解释而受到关注,并逐步被引入到医学影像等对可靠性要求极高的场景中,成为“从像素到结构化表示”的一条新路径。
1.2 存在挑战
传统诊疗方式十分依赖于医生的个人经验,对于大量影像逐张审阅的方式已经难以匹配临床需求。尽管以机器学习和深度学习神经网络为技术核心的医学图像异常检测已在医疗诊断中取得广泛应用,有效提升了诊断效率,但其“检测—分割—诊断”的技术链条仍面临下述阶段性挑战:
检测阶段:小异常在大量背景噪声中易被淹没
医学图像在采集、压缩、传输与储存的过程中,易发生噪声叠加和伪影生成的现象,限制了对组织细节的观察与识别并且干扰特征的提取,会直接影响异常检测的准确性和可靠性。小病灶体积小、对比度低、形态多变,更容易与噪声、伪影混淆。上述退化不仅降低了局部信噪比、削弱边缘与纹理,还会改变图像的强度分布与对比度,使得特征表达变得不稳定,从而在异常检测环节引发阈值敏感、误检与漏检并存等问题。
在实际图像数据中,异常区域在整体图像中占比极低,而正常背景占绝大多数。所以在训练时,模型易被大量“正常区域”—易负样本牵引,出现对多数类过拟合、对少数类欠拟合的现象。同时,对于不同设备间跨设备、跨中心的图像生成风格和细节差异,使得异常检测模型在不同医院和医学研究中心的数据中泛化性差。类别极不平衡与分布差异叠加,常导致模型决策边界偏置、外部验证掉点与稳定性下降;这些因素共同作用,导致了医学图像异常检测技术在小病灶的检测方面的缺陷和不足。
分割阶段:弱对比与边界模糊导致轮廓分割失败
医学图像中,多类病灶与临近组织灰度接近、边界轮廓模糊,在数据上体现为与周围邻近数据梯度低、方差低。成像时考虑到层厚和各向异性的因素,使得现有医学图像异常检测中,像素级分割在弱边界处容易出现“漏抠”或“过抠”的现象。同时,精细的像素级人工标注成本高、主观差异大,并且对于分割的标准和阈值一致性有限,使得模型在训练与评估阶段都易收到标注噪声与样本量不足的双重制约,直接影响后续诊断所依赖的结构化证据—异常病灶图像的分割结果的质量。
弱边界问题还呈现出一组连锁反应:由于边界处梯度与方差偏低,模型对阈值设定与后处理高度敏感,同一张图在不同阈值下可能给出差异显著的分割结果;在小体积病灶上,这种敏感性被进一步放大——哪怕是几个像素的边界偏移,也会带来 Dice 明显不稳与召回率的下降。同时,叠加标注噪声与样本量不足,会让模型在训练期出现收敛不稳的现象,表现为 HD95 升高、小病灶漏检与边界锯齿化等失真[3]。上述因素最终降低了分割结果作为结构化证据进入诊断环节的可靠性,增加了临床复核与纠错的工作量。
诊断阶段:基于深度神经网络学习算法生成的结果可解释性不足
医学图像实际诊断的过程中往往需要多模态、多序列与临床背景信息的综合判断,但跨来源、跨设备数据的风格差异使得信息的综合整合与校准更具挑战。由于目前医学图像异常检测的技术基础是深度学习神经网络,其本身端到端的模型过程具有不可解释性,往往难以给出诊断结果的核查证据,在有限样本和外部数据条件下容易出现泛化不稳定、校准偏差的问题。在医学诊断领域,对于方法的鲁棒性和泛化性一剂证据化输出有很高的需求和要求。
2. 研究内容
粒球图(Granular-Ball Graph)是一种把大量像素数据,先自适应聚成若干个区域块——“粒球”,再把这些粒球按区域重叠生成边界,进行连接,生成新图进而进行深度学习的表示技术。粒球图技术会将同质区域自动合成大球,而细节、边界、可疑处会细化成小球,用一个质量阈值 T作为像素合并的标准,是一个具有自适应优势的数据降维的技术。每一个粒球作为一个节点包含了其位置、尺度、均值、方差、梯度、极值等数据,随后用图神经网络进行表征与读出。相比像素级方法,它单元更少、计算复杂度显著低,同时因关键差异处进行“细节处理”——“小粒球”而无关相同处进行“粗处理”——更大区域粒球,从而达到更抗噪,更稳定的效果;在成图过程中,不同粒球的边界区域的生成过程保留了重合信息,在诊断决策的过程中可以自然的标记和表出异常粒球及其边界信息,使得其可解释性更好。此外,旋转、翻转、上下采样等数据增广可直接在图坐标完成,避免反复重建图,训练更高效、复用性更强,从而提高了其鲁棒性和泛化能力。
【图2.1】粒球图关键技术思路展示图
粒球图以其对图像和像素独特的处理方法,为解决医学图像异常诊断的三个阶段问题上带来了新的启发和解决思路。粒球图技术具有自适应多粒度的优势,在处理医学图像时,背景自动合成大球,可疑处和异常处细化,保留细节成为小球,使得小异常不再被大背景稀释,可以针对性解决检测阶段“小病灶被易噪声淹没”的问题,同时粒球图相邻小球间的一致性约束,以及其在每个粒球节点上所做的原型距离评分可以提升异常检测对稀有异常的敏感度并且降低噪声带来的散点误报。对于分割阶段“弱对比、若便捷导致轮廓识别、对齐失败的问题”,粒球图在边界处用密集小球保留并且强化数据,显式保留重叠边界的过渡关系,可以解决“漏抠”“过抠”的分割问题。在诊断过程中,深度神经网络中的注意力机制可以直接读出并定位哪些粒球是支撑诊断结果的证据,使得结论可复合、可解释、可溯源,同时其进行粒球生成的过程中,自适应多粒度的机制和更少的学习单元与参数,可以降低过拟合的风险,同时提高泛化能力和鲁棒性。同时、粒球图技术在生成粒球的过程中针对其半径和数量以“先粗后细”的逻辑过程自适应的形成粒球,与人类大脑认知事物时“大范围首先的认知机制”匹配并且合同,从根本逻辑基础上具有良好的解释性。所以对于基于粒球图表示学习对医学图像异常检测技术进行改进和优化,是一个具有前景,具有建设性的思路。
2.1 粒球图+高斯原型+马氏距离节点异常度量解决医学图像小病灶被淹没的问题
现有的医学图像异常检测方法中,完成深度学习任务的神经网络通常以像素级数据作为学习样本,导致背景中的大规模易负样本将小体积、低对比的小病灶稀释、淹没。高斯原型,是图像中正常样本在特征空间里的代表,其利用正常样本的均值和协方差来刻画正常样本的特征;马氏距离则度量并刻画了一个点离开高斯原型的距离,相比于欧氏距离,马氏距离会考虑各特征为度的尺度不同和相关性。粒球图技术,将相似像素自适应聚为粒球节点(GRI),提取其位置、尺度、均值、方差、梯度等数据,以“小粒球”保留了微小异常等特征,同时用“大粒球”简化了背景中易负样本的计算权重。粒球图完美弥补了高斯原型与马氏距离技术对异常统计数据的提取与强调,以及对正常易负样本数据的概括,粒球图提供的数据可直接用于构建更加精准的高斯原型并且提高马氏距离的计算效率。通过区域重建,仅用正常样本进行学习高斯原型,以马氏距离对节点打分,可以高效准确的解决小病灶的检测与识别问题。
2.2粒球图+频域多尺度能量边界特征强化法解决医学图像弱对比、弱边界的分割问题
在像素级的图像数据提取过程中,弱对比、弱边界容易被平滑与噪声“抹平”,导致“漏抠”“过抠”的现象频频发生。频域多尺度能量可将结构与细节分离:低频刻画全局形状与区域一致性,高频承载边缘与细节特征。基于此,粒球图+频域多尺度能量边界特征强化法先将相似像素自适应聚为粒球节点,使边界处自然形成密集小粒球、均质区域合并为大粒球;在每个粒球上同时提取位置、尺度、均值、方差、梯度等统计,并叠加多尺度频域能量,再按区域重叠建图。一方面用高频能量+梯度共同强化了真实边缘的小粒球,使低对比边界不易丢失;另一方面以大粒球概括背景并提供对相似数据的掩码,降低了伪边缘响应与,减少了计算复杂度。最终在弱对比场景下更精准的地对齐轮廓、减少“锯齿与断裂”现象的发生。
2.3粒球图+注意力读出的证据可视化方法解决医学图像异常检测可解释性问题
粒球图表示技术,在将多粒球建边、构建成新图的过程中,保留了粒球之间的重叠区域数据。粒球之间的边界不仅能表示空间上的接触关系,还能以统计数据的形式储存组织之间的过渡与连接信息。这种“重叠建边”方式能够更真实、可溯源的反应器官与病灶之间的连接层次,为结果解释提供了可溯源的数据指针。同时,我们在粒球图上引入注意力机制,使学习算法自动聚焦于异常相关的节点与边,并将注意力权重进行反投影至原图形成“证据热图”。这样生成的可视化结果呈现出连贯的结构性子图用于诊断结果的解释与核实,提高了诊断模型的可回溯性、和复用性,同时显著增强了医学图像异常检测技术的临床可解释性。
【图2.2】粒球图主要优势与三条存在挑战的对应解决关系图
3. 研究意义
3.1 理论意义
本项目围绕“基于粒球图表示学习的医学图像异常诊断技术”展开研究,从理论层面探索一种结合粒球计算与医学图像异常检测技术的新方式。粒球图通过自适应地将像素聚为粒球节点,并以重叠区域构建边结构,实现了从像素级到区域级的层级表达。这一结构不仅保留了局部统计与空间信息,还为弱边界、小病灶等复杂结构提供了更鲁棒,更精准的方法基础。结合频域能量、多尺度特征与注意力读出的技术,该方法实现了可解释的异常检测。该研究在理论上拓展了粒球图与粒球计算在实际运用中的落地方向,为医学图像异常检测技术的优化以及可解释性的提高提供了新的框架与思路。
3.2 应用意义
在应用层面,本项目以临床影像智能诊断的核心社会需求,对当前医学图像中存在的标注稀缺、弱对比、小病灶易漏检等问题,提出了轻量、可解释的“检测—分割—诊断”优化方案。通过粒球图的多粒度结构,系统能高效处理不同分辨率的 CT、MRI 等医学影像,实现病灶的精准定位、异常的全面捕获以及结果的可视化。注意力读出的证据图可帮助医生直观解释和溯源诊断依据,提升临床诊断的信任度和诊断效率。该研究成果可应用于医院影像科室的智能筛查、随访监测与图像归档系统,助力医疗影像分析的智能化转型,符合“健康中国 2030”战略对智慧医疗与AI辅助诊断的总体发展方向[a]。
4. 具体目标
4.1对于解决小病灶在医学图像异常检测中被淹没的问题
本项目对于“小病灶在大背景中被淹没”的困难,将粒球图将像素自适应聚为区域节点抑制易负样本的干扰,利用高斯原型和马氏距离在节点层面放大小异常的区分度。通过粒球图与高斯原型、马氏距离相结合的技术达到,在不增加显著算力的前提下提升小病灶的检出与定位稳定性的目标。同时,定量化的目标是使得预期小病灶子集PR-AUC与Recall同步提升。
【图组4.1】现有技术PR- AUC Recall展示(真实实验数据图片)与预期曲线展示 标画
4.2对于弱对比弱边界的异常区域轮廓对齐问题
针对“弱对比、弱边界的轮廓对齐”问题,在特征端融合同步的多尺度、多频域能量以形成边界,粒球图在边上保留重叠区域数据,希望达到弱对比、弱边界条件下的轮廓对齐与分割精准度显著上升,从而降低断裂与边界锯齿等现象,提升Dice、HD95与Boundary F1分数等关键指标的目标。
4.3对于提高医学图像异常检测结果可解释性的问题
面向“结果可解释性”提升,项目以粒球图+注意力读出形成“证据子图”,通过重叠建边保留器官—病灶之间的过渡区,连贯的结构化证据,生成“证据热图”,便于医生复核与追踪溯源。该机制同时服务于临床沟通,增强医学图像异常检测技术在临床诊断应用的可信度,为智能随访与影像归档管理提供可解释的数据基础。
5. 国内外研究现状
5.1粒球图与粒球计算技术的国内外发展与研究分析
信息粒化思想可追溯至 Zadeh(1979)的模糊粒计算,随后粗糙集、商空间、云模型等理论为“多粒度—可解释”范式奠定了基础。近年来,国内学界在此脉络上提出“多粒度粒球计算(Granular-Ball Computing, GBC)”,这是一个用大小自适应的“粒球”而非最细粒度“点”来表示与覆盖样本空间的技术,使学习过程在更贴近人类“先粗后细”的认知机制下获得“高效、鲁棒、可解释”的优势。
【图5.1】粒球技术效果展示(从论文中摘取)
2019年以来,夏书银、王国胤和高新波教授带领以国内团队推进了粒球计算的理论与算法,提出从“以点为输入”的分类器,转向“以粒球为输入”的多粒度学习框架。在粒球图生成方面,提出由标签引导的有序稳定的初始化过程,从而使得结果稳定并且收敛。
同时,国内其他学者,将早期依赖 k-means 的聚类方法进行改进,以划分作为依据,不仅提高了处理效率,还将生成和计算复杂度降至 O(mn),并以单一阈值T控制最小粒度,提高了算法和系统的鲁棒性、解释性以及泛化能力。上述一系列的发展与成果,将“多粒度表示”“计算效率”“可解释边界”统一到一个可工程化的框架之中,并据此扩展到图像表征(GBRI)等应用方向,使中国在粒计算、粒球图表示学习的理论与技术领域获得了巨大的原创优势。
基于粒球计算,国内外学者将原有基于像素的分类器与聚类方法与之结合并改进生成了粒球计算的早期核心方法:
(1)粒球分类器(GB-SVM)
粒球支持向量机分类器将输入样本从“像素点”转化成“像素球”,通过控制球的半径来控制粒度的粗细,以调控支持向量机(SVM)学习数据的精度和范围,并且使得输入数据量远小于原有以像素为输入数据的支持向量机分类器,从而提高了分类器的效率和鲁棒性。
(2)粒球聚类
传统的非凸聚类依赖于样本点间的两两距离参数多且低效,粒球计算采用“粒球划分+质量阈值”的方法,以粒球的半径和数量作为唯二的参数,达到了降参、降低计算复杂度和空间复杂度的效果。其半径的自适应特性,避免了传统学习算法在粒球生成中的应用,从而大大降低了粒球生成过程的计算复杂度。
(3)粒球粗糙集
传统粗糙集通常依赖于单一半径或固定邻域,难以自适应连续的数据,同时又带来了很多重复性计算。粒球粗糙集以多粒度粒球族构造近似与简化,以粗粒度球简化并统一易负样本和特征相仿的样本点,细粒度球补充边界区域的数据信息,实现对边界从粗到细的逐级逼近,既减少冗余,又保持了划分分割的准确度。整体上,粒球粗糙集的方法在连续属性、噪声环境与大样本场景中更高效、更稳定,也更易解释。
对比来看,国外长期在“超像素和区域图+GNN和CNN”的工程体系上进行研究和深挖,但受制于单一粒度以及在高算力基础上过分追求像素级精细划分导致的算法和系统性缺陷,比如本文提到的“小病灶被淹没”“弱对比弱边界”“可解释性差”等关键问题。国内以 GBC 为核心提出“自适应多粒度”的统一生成与学习机制,使“结构语义—多尺度—可解释”在同一框架内得到兼顾。二者在目标上具有一致性,都致力于提升图表示质量与图学习性能,但路径不同。
然而,粒球计算在异常图像检测以及专注于医学图像的检测算法中的系统化应用和落地仍然存在缺失和不足。当前粒球计算所完成的代表性工作更多聚焦于图像分类的表征工作以及“压参增效”即用更小的模型、更少的参数实现计算效率的提升以及部署工作的简化。通过对粒球的学习与研究,我们发现其独特的计算方式对解决医学图像异常检测的难点与缺陷具有极大的优势,将粒球图技术和粒球计算应用于医学图像异常检测的“检测—分割—诊断”三个阶段,是粒球计算技术一大具有发展前景的落地与应用方向。
5.2 医学图像异常检测的国内外发展与研究分析
2019 年以来,国内团队围绕“只学正常样本+度量偏离”的弱监督技术方案,推进了医学图像异常检测发展。一方面,引入了自监督预训练与教师—学生蒸馏热图,提高了少标注条件下的异常定位和可视化能力;另一方面,围绕临床高分辨率数据,突破国内算力有限的瓶颈,提出了“轻量、少参数解码器 + 多尺度融合 + 频域增强”的组合方案。上述进展在肝、脑、肺等多部位数据集上在效率与精度的平衡上取得了较为理想的这种结果。逐步形成了以数据定量为主的评测手段,例如Dice、HD95、FROC,提高了医学图像异常检测算法与系统的复用性。
【图5.2】医学图像异常检测主干技术路线(标出粒球图可优化的步骤)
对比来看,国外长期在“生成式与单类学习“GAN+Deep-SVDD”、特征记忆与学生—教师“Student-Teacher”、Transformer与扩散模型等方向持续深度的进行研究挖掘,并借助大规模预训练与跨模态信息的综合构建了较为完善的技术标准。然而,这个研究方向仍严重受制于单一粒度、对高算力环境的依赖,从而产生了下述三个问题。对于下述缺陷,国际研究学者做出了技术上的探索,各种弥补方案的尝试于实践,但仍未能解决以算力依赖为核心的问题根源。
(1)小病灶难检——多尺度与3D融合的代价
针对这个问题,国际研究中常见做法是用3D CNN 编码 + Transformer 全局建模,结合自蒸馏技术提升对微小病灶与跨尺度结构的图像细节数据挖掘。这种方法能细化局部细节、兼顾全局依赖。 但医学影像数据先天具有“高分辨率、像素量大、异常区域体积小”的特征,在小病灶的识别问题上,结果的精确度不足依旧明显。 相比之下,粒球以更少的“粒球节点”替代像素、先粗后细自适应分裂,在节点规模与搜索空间上天然进行数据简化,为小病灶在更低计算下提供结强有力的解决方案与思路。
(2)弱边界、弱对比导致轮廓对齐与分割困难——跨视图与频域增强的局限
为对抗低对比+弱边界,现有的研究有两类的解决方案。一类是跨视图与模态融合的方案,一类引入频域与大感受野,在轻量网络中增强全局结构的识别以及边缘的构建。然而,Transformer与多尺度堆叠在计算复杂度上往往具有很大的不足,同时,对比度与噪声在不同设备间具有普遍差异。与之相比,粒球图用“重叠关系连边+多粒度节点”保留层级与边界数据信息,细粒度使得背景噪声对结果影响更小,在边界模糊,对比度低的场景更利于形成更稳定更精确的分割结果。
(3)可解释性与泛化性差——以神经网络为基础的像素级大算力模型的天然缺陷
目前的医学图像异常检测技术多以深度学习为算法基础。然而其技术基础——神经网络的端到端、过程“黑盒”特性,往往会使结果的可解释性严重降低。临床诊断对于检测与诊断过程的透明与可解释性具有极高的要求。即便给出CAM图像,但仍缺乏更详细、更可信的因果层面解释与证据链。相较之下,粒球图学习可直接在结果中呈现“关键结构子图”,每个粒球作为节点会保存关键的节点信息如梯度、方差、位置等等,为证据热图的生成提供了直接数据基础。其对于边界重叠数据的保留以及利用注意力读出的方案可以形成更清晰的“证据热图”以供医生进行复合和查验。
5.3 总结分析
粒球计算作为国内原创的多粒度学习范式,以自适应“粒球”替代传统像素级样本空间,契合人类“先粗后细”认知机制,在高效性、鲁棒性与可解释性上具备显著优势。国内团队在其理论、算法及应用上持续突破,如构建多粒度学习框架、优化粒球图生成算法,还将其拓展至图像表征等领域,形成了原创技术优势。在医学图像异常检测领域,国内围绕弱监督技术、轻量模型架构等取得进展,相较国外受单一粒度、高算力依赖限制的方案,粒球计算在小病灶检测、弱边界分割及可解释性方面展现出独特价值,为解决医学图像异常检测的关键痛点提供了新路径,未来在该领域的系统化应用与落地值得期待。综上所述,我们将粒球计算、粒球图表示学习技术融入医学图像异常检测系统,进而弥补现有异常检测系统的缺陷与不足,从而得到兼具高效性、精准性、鲁棒性以及可解释性的异常检测方法。
参考文献
[1]中国医疗机构影像科现状调查报告(2024)[R]. 中国医学影像学杂志,2024.
[2] AI医学影像诊断现状报告[R]. 国家卫生健康委员会,2025.
[3] Karimi D, Rollins C K, Velasco-Annis C, Ouaalam A, Gholipour A. Learning to segment fetal brain tissue from noisy annotations [J]. Medical Image Analysis, 2023, 85: 102731.  
[a]健康中国2030规划纲要(2020)[R]. 中共中央、国务院,2020.  \\ \hline
    \\ \hline
 2.项目拟解决的关键科学问题,研究内容、总体框架、重点难点、主要目标(此部分为重点阐述内容)  \\ \hline
    \\ \hline
    \\ \hline
 3.【理工医科】拟采取的研究方案及可行性分析(包括研究方法、技术路线、实验手段、关键技术等说明);【哲学社会学科】思路方法(本课题研究的基本思路、具体研究方法、研究计划及其可行性等)  \\ \hline
    \\ \hline
 4.本项目的特色与创新点(建议2-3点);  \\ \hline
    \\ \hline
\end{tabular}
\end{table}

\begin{table}[H]
\centering
\begin{tabular}{|l|}
\hline
 研究基础与工作条件  \\ \hline
 1.项目负责人研究基础(建议300字以内)  \\ \hline
    \\ \hline
 2.指导教师研究基础(与本项目相关的研究工作积累和已取得的研究工作成绩,建议300字以内);  \\ \hline
    \\ \hline
 3.工作条件(建议200字以内)  \\ \hline
    \\ \hline
\end{tabular}
\end{table}

\begin{table}[H]
\centering
\begin{tabular}{|l|l|}
\hline
 (三)承担的与本项目相关的科研项目情况  \tabularnewline     \\ \hline
 指导教师曾经和正在承担和参加的省部级以上科研和教改项目情况  \tabularnewline  格式如下,有则填,不限条目:
1. 格式:项目类别,批准号,名称,研究起止年月,获资助金额,项目状态(已结题或在研等),主持或参加
例如:国家自然科学基金面上项目,21773999,×××××××××,2018/01-2021/12,30万元,已结题,主持
2. ……  \\ \hline
 负责人之前参与大创项目情况  \tabularnewline  格式如下,有则填,不限条目:
1. 负责人/团队成员,项目编号,项目名称,立项级别,项目类别,立项年份,结题成绩;
2. ……  \\ \hline
 项目成员之前参与大创项目情况  \tabularnewline  1. 姓名,负责人/团队成员,项目编号,项目名称,立项级别,项目类别,立项年份,结题成绩;
2. ……  \\ \hline
\end{tabular}
\end{table}

\begin{table}[H]
\centering
\begin{tabular}{|l|l|}
\hline
 (四)完成大创项目情况(对负责人负责的前一个大创(项目名称及编号)完成情况、后续研究进展及与本申请项目的关系加以详细说明。另附该已结题项目研究工作总结摘要和创新点(限200字)和相关成果的详细目录。未承担过的写“无”)。  \tabularnewline     \\ \hline
 项目名称及编号  \tabularnewline     \\ \hline
 完成情况与后续研究进展  \tabularnewline     \\ \hline
 与本申请项目的关系  \tabularnewline     \\ \hline
 工作总结摘要及创新点  \tabularnewline     \\ \hline
\end{tabular}
\end{table}

\begin{table}[H]
\centering
\begin{tabular}{|l|l|}
\hline
 (五)申请人成果和奖励情况
(请注意:①投稿阶段的论文可以列出;②对期刊论文:应按照论文发表时作者顺序列出全部作者姓名、论文题目、期刊名称、发表年代、卷(期)及起止页码(摘要论文请加以说明);③对会议论文:应按照论文发表时作者顺序列出全部作者姓名、论文题目、会议名称(或会议论文集名称及起止页码)、会议地址、会议时间;④应在论文作者姓名后注明第一/通讯作者情况:所有共同第一作者均加注上标“\#”字样,通讯作者及共同通讯作者均加注上标“*”字样,唯一第一作者且非通讯作者无需加注;⑤所有代表性研究成果和学术奖励中本人姓名加粗显示。)  \tabularnewline     \\ \hline
 1.代表性成果(包括论文、专利、专著、科创竞赛获奖、学术交流活动、奖学金等,限合计5项)  \tabularnewline     \\ \hline
 论文  \tabularnewline  期刊论文
示例
(1) 冯建涛,陈海峰,李良超*,ZnTi0.6Fe1.4O4/膨胀石墨复合物对污染物的吸附-光催化降解活性,中国科学:化学,2015,45(10):1075 \textasciitilde{} 1088
(2) Liming Tan\#, Kelvin Xi Zhang\#, Matthew Y. Pecot, Sonal Nagarkar-Jaiswal, Pei-Tseng Lee, Shin-ya Takemura, Jason M. McEwen, Aljoscha Nern, Shuwa Xu, Wael Tadros, Zhenqing Chen, Kai Zinn, Hugo J. Bellen, Marta Morey*, S. Lawrence Zipursky*, Ig Superfamily Ligand and Receptor Pairs Expressed in Synaptic Partners in Drosophila, Cell, 2015, 163(7): 1756-1769
会议论文
示例:Lou Y.\#, Zhang H., Wu W., Hu Z., Magic view: An optimized ultra-large scientific image viewer for SAGE tiled-display environment, 9th IEEE International Conference on e-Science, e-Science 2013, Beijing, P.R. China, 2013.10.22-10.25  \\ \hline
 专利  \tabularnewline  授权发明专利
格式:发明人,专利名称,授权时间,国别,专利号
示例:王凡, 一种改善营养性贫血的中药组合物及其制备方法,2014.11.19,中国,ZL201210020610.9  \\ \hline
 专著  \tabularnewline  专著
格式:所有作者,专著名称(章节标题),出版社, 总字数,出版年份。
示例:许智宏,种康,植物细胞分化与器官发生,科学出版社,420千字,2015  \\ \hline
 科创竞赛获奖  \tabularnewline     \\ \hline
 学术交流活动  \tabularnewline  会议特邀学术报告
格式:报告人,报告名称,会议名称,会议地址,会议时间
(1) 郑晓静,风沙环境下高雷诺数壁湍流研究,第八届全国流体力学学术会议,中国,兰州,2014年9月18-21日
(2) Jiang Zonglin, Experiments and Development of Long-test-duration Hypervelocity Detonation-driven Shock Tunnel , 2014 AIAA Science and Technology Forum and Exposition, National Harbor, Maryland , 13 - 17 January 2014  \\ \hline
 奖学金  \tabularnewline     \\ \hline
 2.代表性之外成果和奖励(限合计不超过5项)。  \tabularnewline     \\ \hline
 获得奖励
格式:获奖人(获奖人排名/获奖人数),获奖项目名称,奖励机构,奖励类别,奖励等级,颁奖年份(所有获奖人名单附后)
示例:李兰娟(1/15),重症肝病诊治的理论创新与技术突破,国家科技部,国家科学技术进步奖,一等奖,2013
(李兰娟,郑树森,陈智,李君,王英杰,徐凯进,徐骁,陈瑜,刁宏燕,杜维波,王伟林,姚航平,吴健,曹红翠,潘小平)  \tabularnewline     \\ \hline
\end{tabular}
\end{table}

\begin{table}[H]
\centering
\begin{tabular}{|l|l|}
\hline
 (六)预期成果形式(可多选)  \tabularnewline     \\ \hline
 1.□SCI论文  篇  
2.□核心期刊论文  篇  
3.□会议论文  篇  
4.□内部编印期刊论文  篇
5.□授权发明专利  项  
6.□申请发明专利  项   
7.□创新创业类竞赛获奖  
8.□参加国际国内学术交流活动
9.□其他      名称:  \tabularnewline     \\ \hline
    \tabularnewline     \\ \hline
 (七)项目经费概要(按申报项目目标任务需要进行预算,经费执行情况将与结题考核成绩挂钩)  \tabularnewline     \\ \hline
 1.申请经费明细
(1)仪器设备费         
(2)耗材费          
(3)测试加工费          
(4)国内会务及差旅费         
(5)国外会务及差旅费         
(6)文献/知识产权事务费         
(7)办公费(含文印、办公用品等)          
(8)其他费用           
2.合计  \tabularnewline     \\ \hline
\end{tabular}
\end{table}

\textbackslash{}\textbackslash{}

\begin{table}[H]
\centering
\begin{tabular}{|l|}
\hline
 评审情况  \\ \hline
 指导教师意见:  \\ \hline
    \\ \hline
 指导教师(签名):                 年  月  日  \\ \hline
 学院推荐意见:  \\ \hline
    \\ \hline
 主管院长签名:                  年  月  日  \\ \hline
 学校专家评审意见:  \\ \hline
    \\ \hline
 组长签名:                 年  月  日  \\ \hline
 学校认定意见及批准经费:  \\ \hline
    \\ \hline
 学校负责人签名:                年  月  日  \\ \hline
\end{tabular}
\end{table}


\end{document}
