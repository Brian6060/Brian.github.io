\documentclass{mcmthesis}

% —— Contest header (edit as needed) ——
\mcmsetup{
  tstyle=\color{black}\bfseries,
  tcn = 1234,            % TODO: replace with your official team control number
  problem = C,           % TODO: replace with the problem letter
  sheet = true,          % Summary on a separate page
  titleinsheet = true,
  keywordsinsheet = true,
  titlepage = false,
  abstract = true
}

% —— Built-in packages only (no external assets) ——
\usepackage{newtxtext,newtxmath}   % If missing fonts, fall back to: \usepackage{txfonts}
\usepackage{indentfirst}
\usepackage{graphicx}
\usepackage{amsmath,amssymb}
\usepackage{amsthm}
\usepackage{hyperref}
\usepackage{listings}
\usepackage{xcolor}
\usepackage{mwe}                   % Provides example-image placeholders
\usepackage{array}
\usepackage{float}          % 提供 [H]
\usepackage[section]{placeins} % 可选:保证每节内不跨越
\usepackage{longtable}

% —— Code listing style (tweak as needed) ——
\lstset{
  basicstyle=\ttfamily\small,
  numbers=left,
  numberstyle=\tiny,
  frame=single,
  breaklines=true
}

% —— Convenience macros ——
\newcommand{\TODO}[1]{\textcolor{red}{\textbf{TODO:} #1}}
\newcolumntype{L}[1]{>{\raggedright\arraybackslash}p{#1}}

\title{A Ready-to-Use \LaTeX{} Template for \MCMversion}
\author{(Removed per contest rules)}
\date{\today}

\begin{document}

% ========================= Summary Page (≤1 page) =========================
\begin{abstract}
\noindent\textbf{Problem \& Goal.}
\ TODO: Rephrase the problem in 1--2 sentences and state measurable goals; specify scope (who/where/when) and KPIs (e.g., cost, accuracy, coverage).

\smallskip
\noindent\textbf{Method.}
\ TODO: Name the core model/algorithm and key ideas (e.g., conservation law / optimization / simulation); optionally cite the central equation or workflow.

\smallskip
\noindent\textbf{Key Results.}
\ TODO: Report headline results with ``numbers + units'' (scenarios A/B/C with metrics X/Y/Z = \dots); add one sentence comparing to baseline/literature/common sense.

\smallskip
\noindent\textbf{Validation.}
\ TODO: State consistency with historical observations / public benchmarks / expert ratings (error, correlation, or statistical significance).

\smallskip
\noindent\textbf{Implications.}
\ TODO: Provide actionable recommendations (prioritized actions or cost--benefit trade-offs) in one concise sentence.

\begin{keywords}
optimization; simulation; modeling; sensitivity analysis; decision making
\end{keywords}

\begin{figure}[h]
  \centering
  \includegraphics[width=0.55\linewidth]{example-image}
  \caption{Main result at a glance (placeholder; replace with your overview figure)}
  \label{fig:Abstract}
\end{figure}
\end{abstract}

\maketitle
\tableofcontents
\newpage

% ========================= 1. Introduction & Goals =========================


\section{Introduction \& Goals}

\noindent\textbf{Background.}
We consider the effects of [PHENOMENON] over [TIME HORIZON] on [REGION/SYSTEM/STAKEHOLDERS]. 
This study is motivated by [CONTEXT \& SIGNIFICANCE], where prior efforts have focused on [PRIOR FOCUS] yet a gap remains in [GAP/LIMITATION]. 
To make the problem concrete, we ask: (i) [SUB-QUESTION 1]? (ii) [SUB-QUESTION 2]? (iii) [SUB-QUESTION 3]? 
We use data from [DATA SOURCE(S)] collected during [DATA WINDOW] and summarize key variables in [TABLE/FIGURE REF]. 
To stress-test conclusions for decision-makers in [TARGET DOMAIN], we analyze [SCENARIOS/CASES] and report implications under [CONSTRAINTS/ASSUMPTIONS].

\medskip
\noindent\textbf{Goals \& KPIs.}
Our goal is to [DESIGN/BUILD/PROPOSE] a [METHOD/SYSTEM] that: 
(a) [GOAL 1], (b) [GOAL 2], and (c) [GOAL 3]. 
We explicitly \emph{exclude} [OUT-OF-SCOPE 1], [2], [3]. 
We evaluate success using pre-defined thresholds: [METRIC A] $\ge$ [THRESHOLD A], [METRIC B] $\le$ [THRESHOLD B], and [METRIC C] within [RANGE], under [TIME/MEMORY/COMPUTE] constraints. 
The study scope includes [GEOGRAPHIC AREA/POPULATION], [APPLICATION SCENARIO], and the period [START DATE]–[END DATE].

\subsection{Evaluation Metrics}
We quantify performance along accuracy, efficiency, and coverage/cost dimensions:

\paragraph{Point/series accuracy.}
Mean Absolute Error (MAE) and Root Mean Squared Error (RMSE) are defined as
\[
\mathrm{MAE}=\frac{1}{n}\sum_{i=1}^{n}\big|y_i-\hat{y}_i\big|,\qquad
\mathrm{RMSE}=\sqrt{\frac{1}{n}\sum_{i=1}^{n}\big(y_i-\hat{y}_i\big)^2 }.
\]

\paragraph{Classification quality.}
Given true positives (TP), false positives (FP), false negatives (FN),
\[
\mathrm{Precision}=\frac{\mathrm{TP}}{\mathrm{TP}+\mathrm{FP}},\quad
\mathrm{Recall}=\frac{\mathrm{TP}}{\mathrm{TP}+\mathrm{FN}},\quad
\mathrm{F1}=2\cdot\frac{\mathrm{Precision}\cdot \mathrm{Recall}}{\mathrm{Precision}+\mathrm{Recall}}.
\]
We summarize the trade-off via the Precision–Recall Area Under Curve (PR–AUC) computed over the score threshold $\tau\in[0,1]$.

\paragraph{Operational metrics.}
Coverage and cost are computed as
\[
\mathrm{Coverage}=\frac{\text{served units}}{\text{eligible units}},\qquad
\mathrm{Unit\ Cost}=\frac{\text{total cost (currency)}}{\text{output units}},
\]
and we also report throughput (items/s) and latency (s) under [HARDWARE/SETTINGS].

\medskip
\noindent We report all metrics with [CONFIDENCE LEVEL]\% confidence intervals over [NUMBER] independent runs or [FOLDS]-fold cross-validation, and we conduct ablation on [KEY COMPONENTS] to assess robustness.

% -------- Optional: compact metric table (fill targets) --------
\begin{table}[!htbp]
\centering
\small
\begin{tabular}{|l|p{7.5cm}|p{5.8cm}|l|}
\hline
\textbf{Metric} & \textbf{Definition / How computed} & \textbf{Why it matters} & \textbf{Target} \\
\hline
MAE & $\frac{1}{n}\sum_i |y_i-\hat y_i|$; scale-sensitive absolute error. & Penalizes average absolute deviation; robust to outliers. & $\le$ [X] \\
\hline
RMSE & $\sqrt{\frac{1}{n}\sum_i (y_i-\hat y_i)^2}$; square-penalized error. & Emphasizes large errors; useful for safety-critical tasks. & $\le$ [X] \\
\hline
Precision / Recall / F1 & From TP/FP/FN as above, at threshold $\tau$. & Controls false alarms vs. misses; F1 balances both. & $\ge$ [X]/$\ge$ [Y]/$\ge$ [Z] \\
\hline
PR–AUC & Area under Precision–Recall curve over $\tau\in[0,1]$. & Stable under class imbalance; summary of ranking quality. & $\ge$ [X] \\
\hline
Coverage & $\frac{\text{served}}{\text{eligible}}$ over [REGION/TIME]. & Measures service reach / fairness / access. & $\ge$ [X]\% \\
\hline
Unit Cost & Currency per delivered unit. & Budget feasibility and scalability. & $\le$ [\$X] \\
\hline
Latency / Throughput & End-to-end response time / items per second. & Real-time deployability and user experience. & $\le$ [X] s / $\ge$ [Y] \\
\hline
\end{tabular}
\caption{Evaluation metrics, computation, rationale, and targets. Replace bracketed \textit{[X]} with your thresholds.}
\label{tab:metrics}
\end{table}

% =========================
% Mini Glossary (EN–ZH)
% =========================
\subsection*{Mini Glossary (EN-ZH)}
\begin{description}
  \item[scope] spatial/temporal/population limits of the study.
  \item[constraint] limits on time, memory, budget, or policy.
  \item[validation] testing on held-out data or cross-validation.
  \item[benchmark] a standard dataset/method used for comparison.
  \item[ablation] removing components to isolate their effects.
  \item[assumption] a stated condition required by the model.
  \item[sensitivity analysis] vary inputs/params to test robustness.
  \item[robustness] stability of results under perturbations/shifts.
  \item[coverage] proportion of eligible items successfully served.
  \item[latency/throughput] response time / processing rate.
\end{description}

% ========================= 2. Assumptions & Notation =========================
\section{Assumptions \& Notation}
\subsection{Key Assumptions}
\begin{itemize}
  \item \TODO{Assumption 1 (falsifiable and strongly tied to the model).}
  \item \TODO{Assumption 2.}
  \item \TODO{Assumption 3.}
\end{itemize}

\subsection{Notation Table}
\begin{table}[h]
\centering
\begin{tabular}{|L{3cm}|L{9cm}|}
\hline
\textbf{Symbol} & \textbf{Meaning / Unit} \\
\hline
$i,\,j$ & Row / column indices of the Sudoku grid (1--9). \\
$(i,j)$ & Coordinate of a cell (row $i$, column $j$). \\
$B$ & A Sudoku board (grid). \\
$S(B)$ & Sudoku Solution Graph (SSG): candidate digits set attached to each cell of board $B$. \\
\textit{group} & A row, a column, or a $3\times3$ region (box). \\
$X,\,Y,\,Z$ & Cell variables (uppercase italic). \\
$x$ & A digit value ($1\!-\!9$). \\
$(x, X)$ & Assignment: cell $X$ takes value $x$. \\
$X^{?}$ & Candidate set for cell $X$. \\
$P$ & A Sudoku puzzle (board with some given digits). \\
$E(P)$ & Set of empty cells in puzzle $P$. \\
$c(X)=|X^{?}|$ & Choice function: number of candidates of cell $X$. \\
$\tilde{c}(P)$ & Choice histogram (vector) of puzzle $P$. \\
$c_n(P)$ & The $n$-th bin count of $\tilde{c}(P)$. \\
$w(n)$ & Weight function applied to bin $n$ when computing ease. \\
$\mathrm{wef}(P)$ & Weighted ease function for $P$. \\
$\mathrm{wnef}(P)$ & Weighted \emph{normalized} ease function for $P$. \\
$N$ & Number of DLX solutions requested when seeding a generator. \\
DLX & Knuth’s Dancing Links (Algorithm X) solver for exact cover. \\
\hline
$\Delta z$ & Sea-level change due to thermal expansion (cm). \\
$\Delta T$ & Change in global temperature ($^\circ$C). \\
$k$ & Diffusivity coefficient in upwelling–diffusion model. \\
$\mathrm{local}(t)$ & Expected local sea-level rise at year $t$ (cm). \\
$\mathrm{normalized}(t)$ & Global sea-level rise estimate relative to historical rate (cm). \\
$\mathrm{trend}$ & Local historical rate of sea-level change (cm/yr). \\
$\mathrm{global}(t)$ & Model-predicted global sea-level rise at year $t$ (cm). \\
$S_1(t)$ & Temperature-proportional sea-level component, $S_1=\gamma\,\Delta T(t)$. \\
$\gamma$ & Scaling $\displaystyle \gamma=\frac{\Delta S(2100)}{\Delta T(2100)}$. \\
$S_2(t)$ & Heat-transfer component of sea-level rise. \\
$q$ & Heat flux in generic heat-exchanger equation. \\
$U_a$ & Thermal (convective) coefficient in heat-transfer term. \\
$T_1,\,T_2$ & Temperatures of the two media in the heat-exchange model ($^\circ$C). \\
$\alpha,\,\beta$ & Scaling coefficients from integration / calibration of $S_2(t)$. \\
$t_f$ & Final time (year) for the integration window. \\
$\Delta S(t)$ & Total sea-level rise: $\Delta S(t)=0.55\,S_1(t)+0.45\,S_2(t)$ (cm). \\
$\Delta T_{\text{Gr}}$ & Greenland temperature change; $\Delta T_{\text{Gr}}=2.2\,\Delta T_{\text{global}}$. \\
$M$ & Mass balance of an ice sheet (accumulation $-$ ablation). \\
$\mathrm{SLR}$ & Sea-level rise (often used as a shorthand variable/metric). \\
$h$ & Ice-sheet height (km), $h=\dfrac{\mathrm{Volume}_{\mathrm{ice}}}{\mathrm{Surface}_{\mathrm{ice}}}$. \\
$T_a,\,T_l$ & Upper-surface (air) and lower (permafrost) temperatures of ice sheet ($^\circ$C). \\
\hline
\end{tabular}
\caption{Notation and definitions (symbols gathered from the 2008 MCM/ICM solutions volume).}
\label{tab:notation}
\end{table}

% ========================= 3. Data & Sources =========================
\section{Data \& Sources}
\subsection{Datasets and Acquisition}
\TODO{Inventory datasets, time ranges/versions, acquisition method and license (public repositories, agency portals, etc.).}

\subsection{Preprocessing \& Uncertainty}
\TODO{Describe missing-value handling, resampling/alignment, outliers, measurement error assumptions, and possible biases introduced.}

\begin{table}[h]
\centering
\begin{tabular}{|L{3cm}|L{3.2cm}|L{3.2cm}|L{3.2cm}|}
\hline
\textbf{Dataset} & \textbf{Time Range} & \textbf{Source URL/Ref} & \textbf{Notes} \\
\hline
\TODO{Name A} & \TODO{2000--2024} & \TODO{Ref/Link} & \TODO{Resolution/License} \\
\TODO{Name B} & \TODO{...}        & \TODO{...}      & \TODO{...} \\
\hline
\end{tabular}
\caption{Data inventory.}
\label{tab:data}
\end{table}

% ========================= 4. Model / Algorithm =========================
\section{Model / Algorithm}
\subsection{Core Idea}
\TODO{Why this model: mechanistic interpretability / tractability / match to data. Refer to the overview in Fig.~\ref{fig:workflow}.}

\subsection{Mathematical Formulation}
% —— Example: optimization or equation system (replace as needed) ——
We formulate the problem as
\begin{equation}
  \min_{\theta}\ J(\theta) = \sum_{i=1}^{n} \ell\!\big(f(\mathbf{x}_i;\theta),\, y_i\big)
  \quad \text{s.t.}\quad \mathbf{g}(\theta)\le \mathbf{0},\ \mathbf{h}(\theta)=\mathbf{0}.
  \label{eq:model}
\end{equation}
\TODO{Or provide differential/difference equations, constraints, and boundary conditions; include pseudocode/workflow if appropriate.}

\subsection{Complexity \& Interpretability}
\TODO{State time/space complexity order; list key hyperparameters and their meanings; specify interpretability hooks (feature importance, sensitivity paths, etc.).}

\begin{figure}[h]
  \centering
  \includegraphics[width=0.75\linewidth]{example-image-a}
  \caption{Workflow / model components (placeholder)}
  \label{fig:workflow}
\end{figure}

% ========================= 5. Computation / Implementation =========================
\section{Computation / Implementation}
\TODO{Numerics (discretization, convergence/step size, stability), engineering (language/libs/hardware), and acceleration/robustness (pruning, caching, parallelism).}

\begin{lstlisting}[language=Python, caption={Pseudocode or key routine (placeholder)}]
# TODO: Replace with your algorithmic skeleton
for epoch in range(E):
    theta = update(theta, grad(J))
    if converged(theta): break
\end{lstlisting}

% ========================= 6. Results =========================
\section{Results}
\TODO{Present primary figures/tables with quantitative conclusions. Aim for ``number + unit + comparator'' in every headline claim.}
\begin{figure}[h]
  \centering
  \includegraphics[width=0.72\linewidth]{example-image-b}
  \caption{Main quantitative results under scenarios A/B/C (placeholder)}
  \label{fig:results}
\end{figure}

% ========================= 7. Validation =========================
\section{Validation}
\TODO{Check against history / third-party benchmarks / cross sources; report statistical significance or error bars.}
\[
\text{e.g., } \mathrm{MAE} = \TODO{~value},\quad r = \TODO{~value},\quad p<0.05.
\]

% ========================= 8. Sensitivity Analysis =========================
\section{Sensitivity Analysis}
\TODO{Perturb the most influential parameters (\(\pm 10\%\) / multi-scenario), show KPI variation (heatmap or tornado plot), and discuss robustness.}
\begin{figure}[h]
  \centering
  \includegraphics[width=0.72\linewidth]{example-image-c}
  \caption{Sensitivity of KPI to key parameters (placeholder)}
  \label{fig:sensitivity}
\end{figure}

% ========================= 9. Strengths & Weaknesses =========================
\section{Strengths and Weaknesses}
\subsection{Strengths}
\begin{itemize}
  \item \textbf{Generality} — \TODO{Applicable to multiple scenarios; scalable.}
  \item \textbf{Interpretability} — \TODO{Results are explainable and auditable.}
  \item \textbf{Efficiency} — \TODO{Complexity/runtime advantages and resource friendliness.}
\end{itemize}

\subsection{Weaknesses}
\begin{itemize}
  \item \TODO{Data dependence / boundary conditions / extrapolation risks.}
  \item \TODO{Sensitivity to hyperparameters / scenario setup.}
  \item \TODO{Bias from simplifying assumptions.}
\end{itemize}

% ========================= 10. Decision Recommendations =========================
\section{Decision Recommendations}
\TODO{Actionable checklist for policy/engineering/operations with priorities, cost--benefit, and risks.}
\begin{enumerate}
  \item \TODO{Action 1: expected benefit, cost, and prerequisites.}
  \item \TODO{Action 2: ...}
  \item \TODO{Milestones \& evaluation: when/how to accept results.}
\end{enumerate}

% ========================= 11. Conclusion =========================
\section{Conclusion}
\TODO{Answer the original questions in 2--3 short paragraphs; restate contributions and applicability; outline future work. Cross-reference Eq.~(\ref{eq:model}), Fig.~\ref{fig:results}, and Fig.~\ref{fig:sensitivity}.}

% ========================= 12. Appendix (expand as needed) =========================
\section{Appendix}
\subsection*{A. Derivations / Proofs}
\TODO{Move lengthy derivations here to keep the main text concise.}

\subsection*{B. Additional Algorithms / Pseudocode}
\begin{lstlisting}[language=Python, caption={Example (inline)}]
def selection_sort(arr):
    for i in range(len(arr)):
        m = i
        for j in range(i+1, len(arr)):
            if arr[j] < arr[m]:
                m = j
        arr[i], arr[m] = arr[m], arr[i]
    return arr
\end{lstlisting}

\subsection*{C. Extra Figures and Tables}
\begin{figure}[h]
  \centering
  \includegraphics[width=0.65\linewidth]{example-image}
  \caption{Additional visualization (placeholder)}
\end{figure}

% ========================= References =========================
\begin{thebibliography}{99}
\bibitem{knuth} D.~E. Knuth, \textit{The \TeX{}book}, Addison--Wesley, 1984.
\bibitem{lamport} L. Lamport, \textit{\LaTeX: A Document Preparation System}, Addison--Wesley, 1986.
% TODO: Add data sources and references in a consistent format (avoid bare URLs; \url is fine)
\end{thebibliography}

\end{document}