%% This is file `mcmthesis-demo.tex',
%% generated with the docstrip utility.
%%
%% The original source files were:
%%
%% mcmthesis.dtx  (with options: `demo')
%% !Mode:: "TeX:UTF-8"
%% -----------------------------------
%% This is a generated file.
%% 
%% Copyright (C) 2010 -- 2015 by latexstudio
%%       2014 -- 2019 by Liam Huang
%%       2019 -- present by latexstudio.net
%% 
%% License: The LaTeX Project Public License 1.3c
%% 
%% The Current Maintainer of this work is latexstudio.net.
%% 

% 1. 适配CTeX/TeX Live:根据你的环境选择一行
% 若用CTeX套装(Windows),用这行:
%\documentclass[CTeX = true]{mcmthesis}
% 若用TeX Live(Windows/macOS/Linux),用这行:
\documentclass{mcmthesis}

\mcmsetup{
    tstyle=\color{red}\bfseries, % 题号/队号样式
    tcn = 0000, problem = A,    % 队号、参赛题(需修改为你的信息)
    sheet = true, 
    titleinsheet = false, 
    keywordsinsheet = true,
    titlepage = false, 
    abstract = true
}

% 2. 字体适配:替换为所有TeX发行版兼容的字体
% 注释原字体包,改用基础兼容字体
%\usepackage{newtxtext,newtxmath} % 仅TeX Live部分版本支持
\usepackage{times} % 通用Times字体(所有发行版支持)

\usepackage{indentfirst}  % 首行缩进
\usepackage{lipsum}       % 填充文本(测试用)
\usepackage{graphicx}     % 图片支持(确保已加载)
\usepackage{url}          % 超链接支持(确保已加载)

% 3. 处理图片路径:若没有本地图片,替换为LaTeX内置示例图
% (原代码中的mcmthesis-logo、qrcodewechat可替换为示例图)

\title{The \LaTeX{} Template for MCM Version \MCMversion}
\author{
    \small \href{https://www.latexstudio.net/}{%
        \includegraphics[width=7cm]{example-image} % 替换为内置示例图
    }
}
\date{\today}

\begin{document}
\begin{abstract}
\par Use this template to begin typing the first page (summary page) of your electronic report. This
template uses a 12-point Times New Roman font. Submit your paper as an Adobe PDF
electronic file (e.g. 1111111.pdf), typed in English, with a readable font of at least 12-point type. 

Do not include the name of your school, advisor, or team members on this or any page. 

Be sure to change the control number and problem choice above. 

You may delete these instructions as you begin to type your report here.  

\textbf{Follow us @COMAPMath on Twitter or COMAPCHINAOFFICIAL on Weibo for the
most up to date contest information.}

\begin{keywords}
keyword1; keyword2
\end{keywords}
\end{abstract}
\maketitle

\tableofcontents
\newpage

\section{Introduction}
\subsection{What's this all about? What's \LaTeX?}
\LaTeX\ is a document preparation system which uses the \TeX\
typesetting program. It enables you to produce
publication-quality documents with great accuracy and
consistency. \LaTeX\ works on any computer and produces
industry-standard PDF. It is available both in free (open-source)
and commercial implementations. \LaTeX\ can be used for any kind
of document, but it is especially suited to those with complex
structures, repetitive formatting, or notations like
mathematics. Install the software from
\url{www.tug.org/texlive/}.

\subsection{Creating and typesetting your document}
Follow our WeChat official account for more LaTeX materials and information.

\centerline{
    \includegraphics[width=5cm]{example-image} % 替换为内置示例图
}

\subsection{Syntax (how to type \LaTeX\ commands --- these are the rules)}
\lipsum[3]
\begin{itemize}
\item the angular velocity of the bat,
\item the velocity of the ball, and
\item the position of impact along the bat.
\end{itemize}
\lipsum[4]
\emph{center of percussion} [Brody 1986], \lipsum[5]

\begin{Theorem} \label{thm:latex}
\LaTeX
\end{Theorem}
\begin{Lemma} \label{thm:tex}
\TeX .
\end{Lemma}
\begin{proof}
The proof of theorem.
\end{proof}

\subsection{Other Assumptions}
\lipsum[6]
\begin{itemize}
\item 
\item 
\item 
\item 
\end{itemize}
\lipsum[7]

\section{Analysis of the Problem}
\begin{figure}[h]
\small
\centering
\includegraphics[width=8cm]{example-image-a} % 内置示例图(无需本地文件)
\caption{The name of figure} \label{fig:aa}
\end{figure}

\lipsum[8] \eqref{aa}
\begin{equation}
a^2 \label{aa}
\end{equation}

\[
  \begin{pmatrix}{*{20}c}
  {a_{11} } & {a_{12} } & {a_{13} }  \\
  {a_{21} } & {a_{22} } & {a_{23} }  \\
  {a_{31} } & {a_{32} } & {a_{33} }  \\
  \end{pmatrix}
  = \frac{{Opposite}}{{Hypotenuse}}\cos ^{ - 1} \theta \arcsin \theta
\]
\lipsum[9]

\[
  p_{j}=\begin{cases} 0,&\text{if $j$ is odd}\\
  r!\,(-1)^{j/2},&\text{if $j$ is even}
  \end{cases}
\]

\lipsum[10]

\[
  \arcsin \theta  =
  \mathop{{\int\!\!\!\!\!\int\!\!\!\!\!\int}} \limits_\varphi
  {\mathop {\lim }\limits_{x \to \infty } \frac{{n!}}{{r!\left( {n - r}
  \right)!}}} \eqno (1)
\]

\section{Calculating and Simplifying the Model  }
\lipsum[11]

\section{The Model Results}
\lipsum[6]

\section{Validating the Model}
\lipsum[9]

\section{Conclusions}
\lipsum[6]

\section{A Summary}
\lipsum[6]

\section{Evaluate of the Mode}

\section{Strengths and weaknesses}
\lipsum[12]

\subsection{Strengths}
\begin{itemize}
\item \textbf{Applies widely}\\
This  system can be used for many types of airplanes, and it also
solves the interference during  the procedure of the boarding
airplane,as described above we can get to the  optimization
boarding time.We also know that all the service is automate.
\item \textbf{Improve the quality of the airport service}\\
Balancing the cost of the cost and the benefit, it will bring in
more convenient  for airport and passengers.It also saves many
human resources for the airline.
\end{itemize}

\subsection{How to cite?}
bibliography cite use \cite{1,2,3}

AI cite use \AIcite{AI1,AI2,AI3}

\begin{thebibliography}{99}
\bibitem{1} D.~E. KNUTH   The \TeX{}book  the American
Mathematical Society and Addison-Wesley
Publishing Company , 1984-1986.
\bibitem{2}Lamport, Leslie,  \LaTeX{}: `` A Document Preparation System '',
Addison-Wesley Publishing Company, 1986.
\bibitem{3}\url{https://www.latexstudio.net/}
\end{thebibliography}

\begin{appendices}

\section{First appendix}

In addition, your report must include a letter to the Chief Financial Officer (CFO) of the Goodgrant Foundation, Mr. Alpha Chiang, that describes the optimal investment strategy, your modeling approach and major results, and a brief discussion of your proposed concept of a return-on-investment (ROI). This letter should be no more than two pages in length.

\begin{letter}{Dear, Mr. Alpha Chiang}
\lipsum[1-2]
\vspace{\parskip}
Sincerely yours,
Your friends
\end{letter}

% 4. 处理代码文件路径:若没有本地代码,注释或替换为示例代码
Here are simulation programmes we used in our model as follow.\\
\textbf{\textcolor[rgb]{0.98,0.00,0.00}{Input matlab source:}}
% 若没有./code/mcmthesis-matlab1.m,替换为示例代码
\begin{lstlisting}[language=Matlab]
% Example Matlab code
clear; clc;
x = 1:10;
y = x.^2;
plot(x,y);
\end{lstlisting}

\section{Second appendix}
some more text \textcolor[rgb]{0.98,0.00,0.00}{\textbf{Input C++ source:}}
% 若没有./code/mcmthesis-sudoku.cpp,替换为示例代码
\begin{lstlisting}[language=C++]
// Example C++ code
#include <iostream>
using namespace std;
int main() {
    cout << "Hello MCM!" << endl;
    return 0;
}
\end{lstlisting}

\end{appendices}


\AImatter
\begin{ReportAiUse}{9}
\bibitem{AI1}
Bing AI\\
Query1: write a sort algorithm\\
Output: Sure, I can help you with that. Here is a simple implementation of the selection sort algorithm in Python:
\begin{lstlisting}[language=python]
def selection_sort(arr):
    for i in range(len(arr)):
        min_idx = i
        for j in range(i + 1, len(arr)):
            if arr[min_idx] > arr[j]:
                min_idx = j
        arr[i], arr[min_idx] = arr[min_idx], arr[i]
    return arr
\end{lstlisting}
This algorithm sorts an array by repeatedly finding the minimum element from the unsorted part of the array and putting it at the beginning. \href{https://www.geeksforgeeks.org/sorting-algorithms/}{The time complexity of this algorithm is O($n^2$)}.

There are many other sorting algorithms available such as bubble sort, insertion sort, merge sort, quick sort, heap sort, and more. Each of these algorithms has its own advantages and disadvantages. You can learn more about them on GeeksforGeeks and Programiz.

I hope this helps!

\bibitem{AI2}
OpenAl Enie (Nov 5, 2023 version, Ernie 4.0)\\
Query1: <insert the exact wording of any subsequent input into the Al tool>\\
Output: <insert the complete output from the second query>

\bibitem{AI3}
Github CoPilot (Feb 3, 2024 version)\\
Query1: <insert the exact wording you input into the Al tool>\\
Output: <insert the complete output fron the Al tool>

\bibitem{AI4}
Google Bard (Feb 2,2024 version)\\
Query: <insert the exact wording of your query>\\
Output: <irsert the complete output from the Al tool>

\bibitem{AI5}
Bing AI\\
Query: write a FFT algorithm\\
Output: Sure, here is a Python implementation of the Fast Fourier Transform (FFT) algorithm:
\begin{lstlisting}[language=python]
import numpy as np

def fft(x):
    n = len(x)
    if n == 1:
        return x
    even = fft(x[0::2])
    odd = fft(x[1::2])
    T = [np.exp(-2j * np.pi * k / n) * odd[k] for k in range(n // 2)]
    return [even[k] + T[k] for k in range(n // 2)] + [even[k] - T[k] for k in range(n // 2)]
\end{lstlisting}
The FFT is a widely used algorithm for computing the \textbf{Discrete Fourier Transform (DFT)} of a sequence. It reduces the number of computations needed from O(n$^2$) to O(n log n) 12. The basic idea of the FFT is to apply divide and conquer. We divide the coefficient vector of the polynomial into two vectors, recursively compute the DFT for each of them, and combine the results to compute the DFT of the complete polynomial 3.

I hope this helps!

\end{ReportAiUse}

\end{document}
%% 
%% This work consists of these files mcmthesis.dtx,
%%                                   figures/ and
%%                                   code/,
%% and the derived files             mcmthesis.cls,
%%                                   mcmthesis-demo.tex,
%%                                   README,
%%                                   LICENSE,
%%                                   mcmthesis.pdf and
%%                                   mcmthesis-demo.pdf.
%%
%% End of file `mcmthesis-demo.tex'.